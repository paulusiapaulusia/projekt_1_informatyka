\documentclass[10pt,a4paper]{article}

% ------------------------- PREAMBUŁA -------------------   % stała ścieżka względna do katalogu z  obrazkami.
\usepackage{graphicx}
%\input{settings/packages}  
%\graphicspath{{images/}} 
\usepackage{graphicx}
\usepackage{subcaption}
\usepackage[polish]{babel}
\usepackage[T1]{fontenc}
\usepackage[utf8]{inputenc}
\usepackage{amsmath, amsfonts, amssymb}
\usepackage{booktabs}
\usepackage[top=2.5cm, bottom=2.5cm, left=2cm, right=2cm]{geometry}
\usepackage{textcomp}
\usepackage{gensymb}
\usepackage{textgreek}
\usepackage{geometry}
\usepackage{pdflscape}
\usepackage{pdfpages}
\usepackage{hyperref}
\usepackage{xcolor}

% METADATA
% DOCUMENT METADATA
%\newcommand{\logoGIK}{WGiK-znak.png}
\newcommand{\authorName}{Paula Domarecka, Emilia Brytan \\ grupa 1, Numer Indeksu: 325739, 3257}

\newcommand{\titeReport}{Transformacje współrzędnych} % <<< here insert short title in the food
\newcommand{\titleLecture}{Informatyka Geodezyjna II \\ sem. III, ćwiczenia, rok akad. 2023-2024} % <<< insert title of presentation
\newcommand{\kind}{report}
%\newcommand{\mymail}{01169833@pw.edu.pl \\ 01169842@pw.edu.pl}
\newcommand{\supervisor}{....}
\newcommand{\gikweb}{\href{www.gik.pw.edu.pl}{www.gik.pw.edu.pl}}
\newcommand{\institut}{Zakład Geodezji Wyższej i Astronomii}
\newcommand{\faculty}{Wydział Geodezji i Kartografii}
\newcommand{\university}{Politechnika Warszawska}
\newcommand{\city}{Warszawa}
\newcommand{\thisyear}{2024}
%\date{}
% PDF METADATA
\pdfinfo
{
	/Title       (GIK PW)
	/Creator     (TeX)
	/Author      (Imię Nazwisko)
}
\begin{document}
%	\begin{center} 
		\rule{\textwidth}{.5pt} \\
		\vspace{1.0cm}
%		\includegraphics[width=.4\paperwidth]{\logoGIK}
%		\vspace{0.5cm} \\
		\Large \textsc{\titeReport}
		\vspace{0.5cm} \\  
		\large \textsc{\titleLecture}
		\vspace{0.5cm}\\
		\textsc{\authorName}  \\
		\mymail \\
		\textsc{\faculty}, \textsc{\university}  \\ 
		\city, \today
%	\end{center}
	
	\tableofcontents
	\newpage
\section{Wstęp}
\subsection{Cel ćwiczenia}
Ćwiczenie nr 1 polega na napisaniu programu, który wykona poszczególne transformacje. Program ma potrafić transformować wiele współrzędnych zapisanych w pliku tekstowym przekazywanym do programu jako argument i tworzyć plik wynikowy. 
\section{Przbieg ćwiczenia}
\subsection{Wykorzystane narzędzia}
Do napisania programu zostało przez nas wykorzystane:
\begin{itemize}
	\item Python 3.9
	\item System operacyjny Microsoft Windows 11
\end{itemize}


\section{Etapy rozwiązywania}
Krokiem od którego zaczełyśmy pracę było zapoznanie się z zasadami działania git-huba, co zajęło nam bardzo dużo czasu. Problematycznym było porozumiewanie się przez ten program, ponieważ pomimo wykonywania tych samych algorytmów na dwóch komputerach-program odpowiadał inaczej. Być może było to spowodowane procesami zachodzącymi w tle, więc musiałyśmy utworzyć nowe repozytorium i wgrać pliki, nad którymi pracowałyśmy. Kolejnym problemem,m z którym spotkała się nasza grupa było (nieświadome) utworzenie nowej gałęzi i zapisywanie pracy na niej-zamiast na głównej. Spowodowało to, że różne wersje pliku "geo_v1.py" były na dwóch gałeziach-wobec czego wprowadzałyśmy zmiany edytując różne wersje tego samego pliku.
Problem ten został rozwiązany za pomocą poleceń "git checkout main" ---> "git merge master"---> "git push origin main", nie zapominając o ręcznym edytowaniu pliku "geo_v1.py" w notatniku gdzie wspólnie wprowadziłyśmy niezbędne zmiany i połączyłyśmy swoje wersje
Pisanie kodu było mniej komplikowane, gdyż część transformacji miałyśmy na poprzednim semestrze. Przy wprowadzeniu kilku poprawek były one gotowe do użycia. Następnie zabezpieczyłyśmy program przed nieprawidłowym wpisaniem wartości przez użytkownika.


\section{Podsumowanie}
Wiele aspektów z tego ćwiczenia było trudnych, skomplikowanych. Wymagało to od nas doedukowania się oraz uzupełnienia wiedzy z zajęć. Pomocny był fakt, że ćwiczenie zostało przydzielone w parach- mogłyśmy sobie pomagać oraz uzupełnić nawzajem swoją wiedzę. 
 

\section{Link do repozytorium GIT-HUB}
\href{}

	
\end{document}